\documentclass[hyperfer,UTF8,a4paper,12pt]{article}
\usepackage{tikz,lipsum,lmodern}
\usepackage[a4paper,top=4cm,bottom=2cm,left=2cm,right=2cm]{geometry}  
\usepackage{listings} 
\usepackage{amsthm}
\usepackage{amssymb}
\usepackage{amsmath}
\usepackage{setspace}
\usepackage{geometry}
\usepackage{indentfirst}
\usepackage{amsfonts}
\usepackage{ctex}
\usepackage{titlesec}
\usepackage{esvect}
\usepackage{extarrows}
\usepackage{upgreek}
\usepackage{esint}
\usepackage{verbatim}
\usepackage{latexsym}
\usepackage{fancyhdr}
\usepackage[colorlinks,linkcolor=blue]{hyperref}%[hidelinks]可删去标志
\usepackage[all]{xy}
\usepackage{ulem}

\pagestyle{fancy}
\fancyhf{}
\lhead{Li Xiao's Notes}
\chead{\slshape \leftmark}
\rhead{\thepage}

\linespread{1.4}
\setlength{\parskip}{0.2\baselineskip}
\addtolength{\topmargin}{-54pt}
\setlength{\oddsidemargin}{0.63cm}
\setlength{\evensidemargin}{\oddsidemargin}
\setlength{\textwidth}{14.66cm}
\setlength{\textheight}{24.00cm}
\makeatletter
\let\@afterindentfalse\@afterindenttrue
\@afterindenttrue
\makeatother


\theoremstyle{plain}
\newtheorem{Thm}{$\mathcal{T}$heorem}
\newtheorem*{Prf}{$\mathcal{P}$roof}
\newtheorem*{Lem}{$\mathcal{L}$emma}
\newtheorem*{Cor}{$\mathcal{C}$orollary}
\newtheorem*{Prop}{$\mathcal{P}$roposition}
\newtheorem*{Remark}{$\mathcal{R}$emark}
\newtheorem{Def}{{$\mathcal{D}$efinition}}[section]
\newtheorem*{Example}{$\mathcal{E}$xample}
\newtheorem*{Conj}{$\mathcal{C}$onjecture}
\newtheorem*{Exercise}{$\mathcal{E}$xercise}
\newtheorem{Exer}{习题}[section]
\newtheorem*{Background}{{Background}}
\newtheorem*{Setting}{{Setting}}



\title{分析第二学期内容}
\author{李潇}
\begin{document}


\maketitle
\author
\tableofcontents
\newpage
\begin{abstract}
	本课程由梁永祺老师和许金兴老师执教。为中法班培养计划的第二学期课程。
	
	我正在学习\LaTeX 中,学到的新的关于\LaTeX 技巧会在这里尝试,所以看起来会有点奇怪。打出这份大纲一方面为了记录学了哪些内容,另一方面为了锻炼我的\LaTeX 水平。
	
	还有请忽略我拙劣的英语水平,我只能保证我看得懂,不能保证语法一定正确,所有英语都是我凭感觉写的。
	由于正在学习法语,所以单词默认翻译法语,见谅。
\end{abstract}
\section{拓扑(Topology)}
\begin{Remark}
	Click \href{https://en.wikipedia.org/wiki/Topology}{this} to see tpoplogy in wikipedia
\end{Remark}

\subsection{$ \mathbb{R} $上的拓扑}
	\begin{Def}[开集(open/ouvert)]
	称A是开集。
	要么$ A=\emptyset $,要么$ \forall x\in A,\exists \epsilon>0,s.t.(x-\epsilon,x+\epsilon)\subset A $
	\end{Def}


 \begin{Thm}[$ \mathbb{R} $上的开集结构定理]
	设A是$\mathbb{R}$上的开集,则存在至多可数个不交的开区间$(a_i,b_i) $,
	使 $ A=\bigcup\limits_{i=1}^{\infty}(a_i,b_i)$
 \end{Thm}

\begin{Def}[闭集]
 	称$ A\subset \mathbb{R} $是闭集,若$ \complement A $是开的
\end{Def}


\begin{Example}[Cantor三分集]
	$ I_1=[0,1] \quad I_2=[0,\frac{1}{3}]\cup[\frac{2}{3},1]\quad I_2=[0,\frac{1}{9}]\cup[\frac{2}{9},\frac{1}{3}]\cup[\frac{2}{3},\frac{7}{9}]\cup[\frac{8}{9},1]$
	$ I_n=\cdots\quad C=\bigcap\limits_{n=1}^{\infty}I_n $是有限并的无穷交,是闭集 
\end{Example}

\begin{Def}[邻域]
	$ x\in\mathbb{R}$,称$ V $是$ x $的邻域,若$ \exists \mathcal{O} $为开集,s.t. $ x\in\mathcal{O}\subset V $
\end{Def}

\begin{Def}[聚点(accumulation point)]
	$ A\subset\mathbb{R} $,称$ x\in\mathbb{R} $为A的聚点,若$ \forall x$及其邻域$ V $,有$ V\cap A-\{x\}\not=\emptyset $
\end{Def}

\begin{Thm}
	闭集含有自己所有的聚点,反之也对。
\end{Thm}
\begin{Def}[孤立点]
	$ x\in\mathbb{R} $称为孤立点,若$ x\in A $但是x不是A的聚点
\end{Def}
\begin{Def}[距离]
	$ \forall x,y\in \mathbb{R} $ that $ d(x,y)=|x-y|$
\end{Def}
\begin{Def}
	几种连续函数的表达方法,$ f:\mathbb{R}\to\mathbb{R},x\in\mathbb{R} $
	
	\begin{description}
		\item[Level 1] $ \forall \epsilon>0,\exists\delta>0,\forall y\in
	\mathbb{R}\quad $有$\quad |y-x|<\delta\implies|f(x)-f(y)|<\epsilon $
		\item[Level 2]$ \forall \epsilon>0,\exists \delta>0\quad $有$\quad f(B(x,\delta))\subset B(f(x),\epsilon)  $
		\item[Level 3]$ \forall f(x) $的邻域V,$ \exists x $的邻域$ V $,使得$ f(W)\subset V $
	\end{description}
\end{Def}



\subsection{一般的拓扑}
\begin{Def}[拓扑空间]
	一个集合$ X $和一个由X的子集构成的集合$ \mathcal{O} $,$ \mathcal{O}\subset 2^X $ and satisfy that
	\begin{description}
		\item[$ O_1 $] $ \mathcal{O} $中集合的任意并还在$ \mathcal{O} $中 
		\item[$ O_2 $] $ \mathcal{O} $中集合的有限交还在$ \mathcal{O} $中
		\item[$ O_3 $] $ \emptyset $与$ X $在$ \mathcal{O} $中
	\end{description}
	then 称 $ \mathcal{O} $中的元素为开集,$ (E,\mathcal{O} ) $ is a Topology Space
\end{Def}

\begin{Example}[平凡拓扑]
	$ \{X,\mathcal{O}=\{\emptyset,X \}  \} $
\end{Example}
\begin{Example}[离散拓扑]
	$ \{X,\mathcal{O}=2^X \} $
\end{Example}
\begin{Example}
	在$ \mathbb{R} $下$ \mathcal{O}=\{(-\infty,a)|a\in\mathbb{R}\}\cup\{\emptyset,\mathbb{R}\} $
\end{Example}

\begin{Def}[闭集]
	称$A\subset E $是闭集,若$ \complement_{E}A $开
\end{Def}

\begin{Def}[邻域]
	$ x\in E $,称V为x的邻域,若$ \exists $开集$ \mathcal{O} $,s.t.$ x\in\mathcal{O}\subset V $
\end{Def}

\begin{Def}[一点处的邻域基]
	$ \mathcal{V} $是x的所有邻域的集合,称$ \beta\subset\mathcal{V} $是x处的邻域基,若$ \forall V\in\mathcal{V},\exists W\in\beta $s.t.$ W\subset V $
\end{Def}

\begin{Def}[拓扑基]
	一族开集$ \{ \omega_i \}_{i\in I} $是拓扑基,若满足以下两个等价条件之一
	\begin{description}
		\item[(1)] $\forall x\in\mathbb{X},\exists $a subset of $ \{ \omega_i \}_{i\in I}$ to be a 邻域基 of x
		
		\item[(2)]  X中的任何开集,必可写为$ \{ \omega_i \}_{i\in I}$ 中某些元素的并。
	\end{description}
\end{Def}

\begin{Def}[生成的拓扑]
	对于集合$ X $和$ \beta\subset 2^X $,生成的拓扑是包含$ \beta $的最小的拓扑。(需证明存在性和唯一性)
	\begin{Remark}
		actually,$ \tau=\{ \bigcup\limits_{\alpha\in\Lambda}\bigcap\limits_{i\in I_\alpha}\omega_i|I_\alpha<+\infty,\omega_i\in\beta \}\cup\{ \emptyset,X \} $
	\end{Remark}
\end{Def}

\begin{Def}[Point adhérent]
	$ A\subset X,x\in X $,x称为A的adhérent点,若$ \forall V $为x的邻域,有$ V\cap A\not=\emptyset $
\end{Def}

\begin{Def}[聚点]
	$ A\subset X,x\in X $称为A的聚点,若$ \forall V $是x的邻域,$ V\cap A-\{x\}\not=\emptyset $
\end{Def}

\begin{Def}[孤立点]
	$ x\in A $且x不是聚点。
	\begin{Remark}
		$ \exists V\ is\ the\ Nbd\ of\ x $ 使得$V\cap A=\{x\}$
	\end{Remark}
\end{Def}
\begin{Def}[闭包]
	for $ A\subset X $,$\bar{A}$ is the smallest closed set which contains A,包含A的最小闭集
\end{Def}
\begin{Remark}
	$\bar{A}=\bigcap\limits_{\begin{subarray}
		
		B \supset A   \\
		B\textrm{是闭的}
		\end{subarray}}B  $
\end{Remark}
\begin{Def}[导集]
	$ A'=\{ A\textrm{的聚点} \} $
\end{Def}

\begin{Def}[集合的内部]
	包含于A的最大开集
\end{Def}
\begin{Def}[集合的边界]
	$x\in A^* \Leftrightarrow \forall x $的邻域V,$ V\cap A\not=\emptyset$ and $ V\cap\complement A\not=\emptyset $
\end{Def}
\subsection{连续映射与同胚}
\begin{Def}[逐点连续]
	对于$ f:X\to Y $,$ X,Y $是拓扑空间,称$ f $在$ x_0 $处连续,若$ \forall$ V is a Nbd of $ f(x_0) $,$ \exists $ W is a Nbd of $ x_0 $,s.t.$ f(W)\subset V $
\end{Def}
\begin{Def}[连续]
	for all x in X,f is continuous on x,then f is called continuous on X.
\end{Def}
\begin{Remark}
	f is continuous on $ A\subset X $,but we can't say $ f|_A:A\to Y $ is continuous!
\end{Remark}

\begin{Thm}
	对于$ f:X\to Y $,$ X,Y $是拓扑空间,以下等价。
	\begin{description}
		\item[(1)] f连续
		\item[(2)] $\forall $开集$ \mathcal{O}\subset Y $,$ f^{-1}(\mathcal{O})$ 是开集
		\item[(3)] $\forall $闭集$ \mathcal{F}\subset Y $,$ f^{-1}(\mathcal{F})$ 是闭集
		\item[(4)] $ \forall A\subset X ,f(\bar{A})\subset\overline{f(A)}$
	\end{description}
\end{Thm}
\begin{Prop}
	连续函数的复合还是连续函数
\end{Prop}

\begin{Def}[\textbf{同胚}]
	$ f:X\to Y $是连续,且$ \exists g:Y\to X $连续,s.t. $ f\circ g=g\circ f=id $,则称$ f $是X到Y的同胚,$ g $是$ f $的逆。
\end{Def}

\begin{Remark}
	保持开集的双射,拓扑空间的同构
	\[  (X,\mathcal{O}_X)\leftrightarrow(Y,\mathcal{O}_Y)   \]
\end{Remark}

\begin{Def}[拓扑的粗细]
	对于X有拓扑$ \tau_1 $,$\tau_2$,若$ \tau_1\subset\tau_2 $,则称$ \tau_1 $比$ \tau_2 $细。
\end{Def}

\subsection{拓扑下的极限}

\begin{Def}[Hausdorff性质,T2分离性质,Séparé]
	拓扑空间E称为是séparé,若$ \forall x,y\in E $ $ x\not=y $ $ \exists x,y $各自的邻域$ V,W $,且$ V\cap W=\emptyset $
\end{Def}

\begin{Thm}
	E是Hausdorff的,则E中的序列至多有一个极限
\end{Thm}

\begin{Example}[$ \mathbb{R} $中的余可数拓扑]
	没有分离性质,所以序列会有多个极限
\end{Example}

\begin{Remark}
	对于$ x\in\bar{A} $与$ x $是A中元素的极限之间的关系,可参考
	\begin{quote}
		\href{wikipedia.org}{Baire纲性定理}
	\end{quote}
\end{Remark}

\begin{Def}[序列的adhérent point]
	for $(a_n)_{n\in\mathbb{N}} $,若$ a\in E $,称为adhérent point,若$ \forall  $a的邻域V,$ \exists i_0 $,s.t.$ \forall i>i_0 $,$ a_i\in V $
\end{Def}
\begin{Remark}
	可以理解为子列极限
\end{Remark}
\begin{Prop}
	A是$(a_n)_{n\in\mathbb{N}} $的所有adhérent point,则$ A $闭,且$ A=\bigcap\limits^{\infty}_{n=1} \bar{A_n} $,其中$ A_n=\{ a_i|i\geq n \} $
\end{Prop}

\begin{Exercise}
	Set of velours d'adhérence 是闭集,那么是否$ \mathbb{R} $中所有闭集都可以写成某个数列的附着点集?
\end{Exercise}

\begin{Def}[滤子基(base de filtre)与极限过程]
	称集合E的子集$ \beta $为滤子基,若$ \forall B_1,B_2\in\beta\quad B_3\in\beta $s.t.$ B_3\subset B_1\cap B_2 $,且$ \beta $中的元素均非空集。
\end{Def}
\begin{Example}
	$ E=\mathbb{R} ,\beta=\{ (x,+\infty)|x\in\mathbb{R} \}$
\end{Example}
\begin{Example}
		$ E=\{(\pi,\epsilon)\} $分割及选点,$ \beta=\{(\pi,\epsilon)|\ ||\pi||<\epsilon  \} $
\end{Example}

\begin{Def}[\textcolor{red}{沿滤子基的收敛}]
	$ f:X\to Y $,Y is a Topo-Space,$ \beta $ is a base de filtre on X........
\end{Def}


\subsection{乘积拓扑}
\begin{Def}[乘积拓扑抽象定义]
	在两个拓扑空间$ E_1,E_2 $上,要定义$ E_1\times E_2 $上的拓扑。
	\begin{align*}
		&\pi_1: E_1\times E_2 \to E_1 \\
		&\pi_2: E_1\times E_2 \to E_2
	\end{align*}
	称$ E_1\times E_2 $上使$\pi_1,\pi_2$连续的最粗的拓扑为乘积拓扑
\end{Def}
\begin{Def}[乘积拓扑构造定义]
	由形如$ \omega_1\times\omega_2 $的初等开集生成的拓扑称为乘积拓扑
\end{Def}
\begin{Def}[开映射]
	开集的像是开集
\end{Def}
\begin{Prop}
	$ E_1,E_2 $是两个拓扑空间,$ A_1\subset E_1\quad A_2\subset E_2 \quad$,$A_1\times A_2$作为$ A_1,A_2 $的乘积拓扑和作为$ E_1\times E_2 $的子空间拓扑是一样的
\end{Prop}
\begin{Prop}[结合率]
	$ (E_1\times E_2)\times E_3=E_1\times(E_2\times E_3) $
\end{Prop}
\begin{Thm}
	$ X,E_1,E_2 $ is topology space,and $ f:X\to E_1\times E_2 $,then $ f $连续$ \Leftrightarrow $ $\pi_1\circ f  $与$\pi_2\circ f$均连续 
\end{Thm}
\begin{Prop}
	$ E_1,E_2 $ is Separe(Hausdorff),then $E_1\times E_2$ is Hausdorff
\end{Prop}

\subsection{紧性}
\begin{Def}[紧compact]
	称拓扑空间$ E $是紧的,若由开覆盖,则有有限子覆盖。i.e.$ E\subset\bigcup\limits_{\alpha\in I}\mathcal{O}_\alpha\quad \mathcal{O}\alpha$是开集,then $ \exists \alpha_1,\cdots,\alpha_m $s.t. $ E\subset\bigcup\limits_{i=1}^m \mathcal{O}_{\alpha_i} $
\end{Def}
\begin{Remark}
	法语书中Compact=有限覆盖+Hausdorff,Quasi-Compact=有限覆盖
\end{Remark}

\begin{Def}[列紧(Sequentially Compact)]
	E的任何点列有收敛子列
\end{Def}

\begin{Def}[极限点紧(Limit Point Compact)]
	任何无穷点集有极限点
\end{Def}

\begin{Prop}
	紧集的闭子集是紧集
\end{Prop}
\begin{Prop}
	紧集在连续映射下的像是紧集
\end{Prop}
\begin{Prop}
	分离集的紧子集是闭集
\end{Prop}

\begin{Prop}
	$ f:X\to\mathbb{R} $连续,X紧$ \implies $$ f $有界\& 取到最值
\end{Prop}

\begin{Thm}
	若$ (X,d) $是度量空间,则紧$ \Leftrightarrow $列紧
\end{Thm}

\begin{Prop}
	$ \mathbb{R}^n $中,有界闭集$ \Leftrightarrow $ 紧集
\end{Prop}

\subsection{连通性}
\begin{Def}[分解]
	E是一个Topo空间,$ X,Y\subset E $,$ X\sqcup Y=E $,则称X,Y是一个分解
\end{Def}

\begin{Def}[连通]
	若不存在非空开分解,则称E是连通的
\end{Def}

\begin{Prop}
	拓扑空间E,$ X\subset E $,$ A\subset E$连通,若$ A\cap X\not=\emptyset\ A\cap\complement X\not=\emptyset $,则$ A\cap X^*\not=\emptyset $
\end{Prop}

\begin{Thm}
	设$ (A_\alpha)_{\aleph\in I} $是一族E中的连通集,若$ \bigcap\limits_{\alpha\in I}A_{\alpha}\not=\emptyset $则$  \bigcup\limits_{\alpha\in I}A_{\alpha} $连通
\end{Thm}

\begin{Thm}
	A连通,若$ A\subset B\subset \bar{A} $,则B连通
\end{Thm}

\begin{Example}
	$\Gamma: f(x)=sin\frac{1}{x},x\in \left(0,1\right] $,则$ \Gamma\cup\{0\}\times\left[0,1\right] $连通
\end{Example}

\begin{Prop}
	X 连通,$ f:X\to Y $连续,则$f(x)$的图像是连通的
\end{Prop}

\begin{Def}[区域domain]
	连通开集
\end{Def}

\begin{Def}[连通分支]
	$ x\in E $,C(x)为所有含x的连通集的并。
\end{Def}

\begin{Def}[局部连通]
	称E在x上局部连通,若存在一族连通集作为x点的邻域基
\end{Def}

\begin{Def}[道路连通]
	称E是path connected,若$ \forall a,b\in E $,$\exists$连续$ f:[0,1]\to E $,$ f(0)=a,f(1)=b $
\end{Def}

\begin{Thm}
	道路连通$\implies$连通
\end{Thm}

\subsection{分离性、可数性、Urysohn引理}
\begin{Def}[分离性]
	\hspace*{\fill}
	\begin{description}
		\item[T1]$ \forall x\not=y $,$\exists V $是x的邻域,s.t.$ y\not\in V$
		\item[T2]$ \forall x\not=y $,$\exists V $是x的邻域,$\exists W $是y的邻域,s.t.$ V\cap W=\emptyset$
		\item[T3]$ \forall x\in E $,闭集$ A\subset E $,有$ x\in V\quad A\subset W $, s.t.$ V\cap W=\emptyset$
		\item[T4] 闭集$ A,B\subset E $,$ A\cap B\not=\emptyset $,$ \exists  $分别的邻域$ U,V $ s.t.$ U\cap V=\emptyset$
	\end{description}
\end{Def}
\begin{Prop}
	度量空间是T1,T2,T3,T4空间
\end{Prop}

\begin{Thm}
	紧的Hausdorff$ \implies $ T4
\end{Thm}
\begin{Def}[可数性]
	\hspace*{\fill}
	\begin{description}
		\item[可分的(Seperable)] E有可数稠密集
		\item[第一可数(C1)]	$ \forall x\in E $,有可数邻域基
		\item[第二可数(C2)]	E有可数开集基
	\end{description}
\end{Def}

\begin{Prop}
	$ C1\implies C2\implies  $可分
\end{Prop}

\begin{Thm}
	Seperable+度量空间$ \implies $ C2
\end{Thm}

\begin{Thm}[Urysohn引理]
	设E是T4,设A,B是其中不交闭集,则存在连续函数$ f:E\to\mathbb{R} $,使$ f|_A\equiv 0,f|_B\equiv 1$
\end{Thm}

\begin{Thm}[Tietze扩张引理]
	E是T4,$ A\subset E$是闭集,$ f:A\to\mathbb{R} $连续,则存在$ \tilde{f}:E\to\mathbb{R} $连续,且$ \tilde{f}|_A=f $
\end{Thm}

\begin{Lem}
	X,$ A_1,A_2$闭子集,且$ A_1\cup A_2=X $,则
	$ f|_{A_1} $和$ f|_{A_2} $均连续
	$\implies $f在X上连续
\end{Lem}

\section{赋范向量空间(Normed Vector Space)}
\subsection{赋范向量空间}
\begin{Setting}
	$ \mathbb{K}=\mathbb{R},\mathbb{C} $
\end{Setting}
\begin{Def}[范数]
	E is a $ \mathbb{K} $-vecteur space,$ N:E\to\mathbb{R}^+ $,称N是一个范数,若
	\begin{description}
		\item[齐次性]  $ \forall (E,\lambda)\in E\times\mathbb{K}\quad N(\lambda x)=\lambda N(x) $
		\item[三角不等式]$ N(x+y)\leq N(x)+N(y) $
		\item[分离性] $ N(x)=0\implies x=0 $
	\end{description}
\end{Def}

\begin{Def}[内积]
	双线性型$ E\times E\to\mathbb{R} $且满足对称性和正定性
\end{Def}
\begin{Prop}
	内积诱导范数,范数诱导距离
\end{Prop}

\begin{Def}[完备的]
	称度量空间$ (E,d) $是完备的,若任何柯西列在E中收敛
\end{Def}

\begin{Prop}
	X是紧拓扑空间,$ (C^0(X,\mathbb{R}),||\cdot||_\infty) $是完备的
\end{Prop}

\begin{Def}[赋范向量子空间]
	F称为$ (E,||\cdot||) $的子空间,若F是E的向量子空间,且$ (F,||\cdot||) $也是赋范向量空间
\end{Def}

\begin{Def}[乘积空间]
	E,F是两个赋范向量空间,他们的乘积空间为\[ (E\times F,||(x,y)||=max\{||x||_E,||y||_F \}) \]
\end{Def}

\begin{Def}[控制和可忽略]
	设$ (a_n)_{n\in\mathbb{N}} $,$ (b_n)_{n\in\mathbb{N}} $是赋范向量空间E中的两个点列,称a被b控制,若$ \exists C ,\forall n\in\mathbb{N}  ||a_n||\leq C||b_n||$,记作$ a=O(b) $

	称a关于b可忽略,若$ \forall \epsilon\ge 0,\exists N\in\mathbb{N},\forall n\ge N,||a_n||\leq\epsilon ||b_n|| $,记为$ a=o(b) $
\end{Def}

\begin{Def}[模的等价]
	称$ N_1,N_2 $是等价的,若他们可以相互控制
\end{Def}
\begin{Remark}
	等价的模$ \Leftrightarrow $诱导相同的拓扑
\end{Remark}


\begin{Thm}
	$ \mathbb{K}^n $上所有模等价
\end{Thm}

\subsection{赋范向量空间上的函数}

\begin{Def}[]
	$ f:A\to F $,分类定义f在a点的极限为l,$ a\in\overline{\mathbb{R}},l\in\overline{\mathbb{R}} $
\end{Def}

\begin{Remark}
	不紧的拓扑$\to$紧的拓扑,用来定义极限
\end{Remark}

\begin{Prop}[连续延拓]
	$ f:A\to F $,F在A上连续,若$ a\in\overline{A}\setminus A $,若f在a有极限l,则$ \tilde{f}:A\cup\{a\}\to F $连续
\end{Prop}

\begin{Def}[函数极限的比较]
	$ E,F  $ is normed vector space,A$\subset$E,a is an adherent point,$ f,g:A\to F $,$ f={}_a O(g) $ and $ f={}_a o(g)$,mean f被g控制,f关于g可忽略
\end{Def}

\begin{Prop}[极限的运算]
	线性性,标量积,倒数的极限,复合的极限
\end{Prop}

\subsection{紧集上的连续函数}
\begin{Def}[一致连续]
	$ f:(X,d)\to(F,\tilde{d}) $一致连续,$ \forall\epsilon\ge0\exists\delta\ge0\forall x\in A\ \forall y\in B(x,\delta)\cap A \quad ||f(x)-f(y)||\le\epsilon$
\end{Def}

\begin{Thm}[紧集上的连续是一致连续]
	$ A\subset E $是紧集,$ f:A\to F $连续,则$ f $一致连续
\end{Thm}

\begin{Thm}
	$ f:E\to F $线性,则$ f $连续$\Leftrightarrow$$ \exists C>0 $s.t.$ \forall x\in E,\|f(x)\|\leq C\|x\|  $
\end{Thm}

\begin{Prop}
	若E有限维,则任何线性映射连续(无穷维不对!)
\end{Prop}

\begin{Prop}
	多线性映射$ f:E_1\times E_2\times\cdots\times E_p\to \mathbb{F} $,$ E_i $均为有限维的,则f连续,且$ \|f(x_1,\cdots,x_p)\|\leqslant\|x_1\|\cdots\|x_p\| $
\end{Prop}

\begin{Def}[norme sous-multiplication]
	称模$ \|\cdot\| $ est sous-multiplication,如果$ \forall x,y\in E $,$ \|x\cdot y\|\leqslant\|x\|\|y\| $
\end{Def}

\begin{Thm}
	有限维赋范代数,有norme sous-multipilication
\end{Thm}
\begin{Prf}
	since $ \|x\times y\|\leqslant C\|x\|\|y\| $,定义$ N(x)=C\|x\| $即可。
\end{Prf}

\begin{Cor}
	$ M_n(\mathbb{K}) $上有算子模
\end{Cor}


\section{级数}

\subsection{数项级数}
    1.正项级数的收敛

    2.交错级数

    3.级数的结合性与交换性

    4.级数的积


\subsection{NVS级数}
    1.Thm:有限维的$\mathbb{K}$-NVS在任意模下完备

    2.部分和与余项,telescope

    3.Thm:E是有限维的NVS,若$\sum\|u_n\Vert$收敛,则$ \sum u_n $收敛

    4.定义矩阵的$\quad exp(A)=\sum\limits_{n\ge 0} \frac{1}{n!}A^n$和$\quad \frac{1}{I-A}=\sum\limits_{n\ge 0}A^n $


\subsection{关于收敛速度}
1.$ \sum u_n\ \sum v_n $均收敛,且$ v_n $非负则
\[\begin{cases}
 \textrm{若}\quad u_n=O(v_n) \quad\textrm{则} \ R_n(u)=O(R_n(v))  \\

 \textrm{若}\quad u_n=o(v_n) \quad\textrm{则} \ R_n(u)=o(R_n(v)) \\

 \textrm{若}\quad u_n\sim v_n\ \ \qquad\textrm{则} \ R_n(u)\sim R_n(v)
\end{cases}
\]
2.$ \sum u_n\ \sum v_n $均发散,且$ v_n $非负则
$\begin{cases}
\textrm{若}\quad u\,_n=O(v_n)   \quad\textrm{则}  \ S_n(u)=O(S_n(v))\\
\textrm{若}\quad u_n=o(v_n)     \quad\textrm{则}  \ S_n(u)=o(S_n(v))\\
\textrm{若}\quad u_n\sim v_n\ \ \qquad\textrm{则} \ S_n(u)\sim S_n(v)\\
\end{cases}$

\subsection{回到数项级数}
1.定义$ (a_i)_{i\in I}\quad a_i\in \mathbb{R}^+ $为\textbf{可和}的,若对,则$ \{\sum\limits_{i\in J} a_i|\forall J\subset I\textrm{有限子集}\}$有界

2.在$ \mathbb{R} $上可和,即正部和负部分别可和,在$ \mathbb{C} $上可和,即实部和虚部分别可和

\textbf{注}:若$ (a_i)_{i\in I} $为可和的,则I只有可数个不为0,故不妨设I为$\mathbb{N}$

3.Thm:若$ (a_i)_{i\in I} \quad (b_i)_{i\in J} $均为可和的,则$ (\sum\limits_{i\in I} a_i )(\sum\limits_{i\in J} b_i )=\sum\limits_{(i,j)\in I\times J} a_ib_j $

4.Dirichlet积:$\sum a_n\ \sum b_n$绝对收敛,$ C_n=\sum\limits_{d|n} a_db_{\frac{n}{d}}$
则有$ \sum C_n $绝对收敛,且$ (\sum\limits_{n=1}^{\infty} a_n )
(\sum\limits_{n=1}^{\infty} b_n )=
\sum\limits_{n=1}^{\infty} C_n $


5.N(n)表示n的因子个数,当$ \alpha >1$时,$ \sum\limits_{n=1}^{\infty}\frac{N(n)}{n^\alpha}=
 \sum\limits_{n=1}^{\infty}\sum\limits_{p\cdot q=n}\frac{1}{p^\alpha}\cdot\frac{1}{q^\alpha}
 =\zeta(\alpha)^2$

6.$ \varphi(n) $是欧拉函数,当$ \alpha >2$时,$ \sum\limits_{n=1}^{\infty}\frac{\varphi(n)}{n^\alpha}=\frac{\zeta(\alpha-1)}{\zeta(\alpha)} $

$ \sum\limits_{n=1}^{\infty}\frac{\varphi(n)}{n^\alpha}
\sum\limits_{n=1}^{\infty}\frac{1}{n^\alpha}= 
\sum\limits_{n=1}^{\infty}\sum\limits_{n=1}^{\infty}\frac{\varphi(n)}{m^\alpha\cdot n^\alpha}
=\sum\limits_{w=1}^{\infty}\sum\limits_{n|w}\frac{\varphi(n)}{w^\alpha}=
\sum\limits_{w=1}^{\infty}\frac{w}{w^\alpha}=\zeta(\alpha-1)$

7.$ (p_n){n\in\mathbb{N}} $为素数列,定义$ A_N=\{n\in\mathbb{N},n\textrm{的素因子属于}\{p_1,\cdots p_N\} \} $,有$ A_1\subset A_2\subset\cdots $,$ \bigcup\limits_{n=1}^{\infty}A_n=\mathbb{N} $

$\prod\limits_{i=1}^{N}\frac{1}{1-\frac{1}{p^\alpha}}=
\sum\limits_{n\in A_N}\frac{1}{n^\alpha}$,进而有$ \prod\limits_{i=1}^{\infty}\dfrac{1}{1-\frac{1}{p^\alpha}}=\zeta(\alpha) $

\subsection{函数列}\hspace*{-2em}
1.简单收敛(pointwise convergence),逐点收敛(convergence simple)

2.简单收敛的一些性质:保持加法和序关系,但不能保持连续性、有界性、可积性、可导性(就算可积或可导,也没有性质)

3.一致收敛,内闭(紧)一致收敛,在A所有点附近一致.Thm:所有点附近一致$ \Leftarrow $内闭一致

\textbf{注}:一致收敛可以看作是$ \Vert\cdot\Vert_\infty $下的收敛,因此一致收敛有Cauchy表述

4.Thm:双极限定理(一致收敛与极限号的交换问题):

$ f:A\to F \ F$完备,a是A的adhérent point,$ f_n:A\to F $一致收敛,$ \lim\limits_{a}f_n=\ell_n $则:$ \lim\limits_{n\to\infty}\ell_n$和$\lim\limits_{a}f $存在且相等

5.例:$\zeta(z)$在$ \{z\in \mathbb{C}| Re(z)>1 \} $上内闭一致

6.一致收敛与积分的关系(极限号交换问题):$ f_n:[a,b]\to\mathbb{K} $连续且一致收敛,则$ \lim\limits_{n\to\infty}\int\limits_{[a,b]}f_n=\int\limits_{[a,b]}\lim\limits_{n\to\infty}f_n $

7.一致收敛和求导的关系:(没有简单的关系)

Thm:(对多元微积分也对)$ f_n $简单收敛到$ f $,$ f_n$有连续的导数$ f_n' $,且$ f_n' $一致收敛,则有$ f $可导,而且$f'=\lim\limits_{n\to\infty}f_n'  $

向$ C^K $推广:$ \forall i\in[0,k-1] f_n^{(i)} $简单收敛,$   f_n^{(K)}$一致收敛,则$ f\in C^K $并且$ f^{(i)}(x)=\lim\limits_{n\to\infty}f_n^{(i)}(x) $对固定的x

补充:Peano 曲线:存在连续满映射$ f:[0,1]\to\mathbb{R}^2 $
\subsection{函数项级数}\noindent
1.$ \sum u_n\ u_n:A\to F\ $简单收敛,收敛域

2.一致收敛

3.正规收敛(converger Normale)$ \begin{cases}
u_n \textrm{有界}

\textrm{数项级数}\sum\limits_{n\ge 0}N_\infty(u_n)\textrm{收敛}
\end{cases} $

\textbf{注}:M判别法,优级数,Weiertrass判别法:(数项级数)Thm:$ \sum a_n $数项,正项,收敛,若$|u_n|\le a_n$则$ \sum u_n $收敛

Thm:$ \sum u_n $正规收敛,则$ \sum u_n $一致收敛(证明用一致收敛的Cauchy表述)

例子:$ \zeta(z) $在$ \{z\in \mathbb{C}| Re(z)>1 \} $每点附近都正规收敛

Dirichlet判别法:$ \sum\limits_{n\ge 1} a_n(x) $一致有界,$ b_n(x) $单调一致趋于0

Abel判别法:$\ \sum a_n(x) $一致收敛,$ b_n(x) $单调一致有界,则$ \sum a_n(x)b_n(x)$一致收敛

4.一致收敛级数的性质:保持连续、逐项积分、逐项求导(需要$ \sum u_n' $连续一致收敛)

例子:$ M_n(\mathbb{K})\ NVS\ $:$ exp(tM)=\sum\limits_{n\ge 0}\frac{1}{n!}(tM)^n \ t\in \mathbb{K}$则$ \frac{d}{dt}(exp(tM))=M\cdot exp(tM) $

5.研究$ \sum u_n $的渐进行为
\subsection{大定理}
\begin{Def}[等度连续]
	设X上有一族连续函数$ \{f_\alpha:\alpha\in I \} $称为是等度连续的,若$ \forall\epsilon>0,\exists\delta>0,\forall\alpha\in I,\forall x,y\in X\quad |x-y|<\delta\implies|f_\alpha(x)-f_\alpha(y)|<\epsilon $
\end{Def}\noindent
\textbf{Weierstrass定理}:设$ f $是[0,1]上的连续$ \mathbb{R} $值函数,则$ \forall \epsilon $,存在多项式$ g\in\mathbb{R}[x] $,使$ \Vert f-g \Vert<\epsilon $

\textbf{Stone-Weierstrass定理}:$X紧$,$ \mathfrak{B}\subset C(X)$是一个分离X的代数,则$ \mathfrak{B} $ 在$ (C(X),\Vert\cdot\Vert_\infty) $中稠密

\textbf{Ascoli-Arzela定理}:X是紧的度量空间,若$ (f_n)_{n\in\mathbb{N}} $一致有界且等度连续,则$ f_n $有一致收敛的子列

\section{幂级数}
\begin{Def}[幂级数]
	复数列$ (a_n)_{n\in\mathbb{N}} $对应的幂级数是函数项级数$ \sum\limits u_n $,其中$f:\begin{array}[t]{ccc}\mathbb{C} &\longrightarrow &\mathbb{C}\\ z &\longmapsto&a_nz^n\end{array} $
\end{Def}

\begin{Thm}[Abel引理]
	设$ (a_n)_{n\in\mathbb{N}} $,$ z_0\in\mathbb{C} $,$ \sum a_nz_0^n $有界,则$ \forall z\in\mathbb{C},|z|<|z_0| $,级数$ \sum a_nz^n $绝对收敛
\end{Thm}

\begin{Remark}
	$ D_O(0,r)=\{ z\in\mathbb{C} , |z|<r \} $$ \quad $$ D_F(0,r)=\{ z\in\mathbb{C} , |z|\leqslant r \} $
\end{Remark}
\begin{Def}[收敛半径]
	$R=sup\{ r\in\mathbb{R}^+,(|a_n|r^n)\textrm{有界} \}$
\end{Def}

\begin{Prop}
	$ R^{-1}=\limsup\sqrt[n]{|a_n|} $
\end{Prop}

\begin{Thm}[D'Alenbert判别法]
	if $ \exists q<1 $,s.t. $ \frac{a_{n+1}}{a_n}<q $则收敛
\end{Thm}


\begin{Prop}
	$ \sum a_nz^n $,$ \sum b_nz^n $ 半径分别为$ R_a,R_b $,则$ \sum (a+b)_nz^n $ 半径$ R\geqslant min\{R_a,R_b\} $
\end{Prop}

\begin{Def}[Cauchy乘积]
	$ \sum a_nz^n $,$ \sum b_nz^n $ 半径分别为$ R_a,R_b$,$ c_n=\sum\limits_{k=0}^n a_kb_{n-k} $,称$ \sum c_nz^n $为Cauchy乘积,且$ R_c\geqslant min\{R_a,R_c\} $
\end{Def}

\begin{Prop}[一致收敛]
	$ \sum a_nz^n $ 在$ D_F(0,R) $上内闭一致收敛,对于实数的情况,$ \sum a_nt^n $ 在端点处收敛则必定一致收敛。
\end{Prop}
\begin{Cor}
	设$ \sum a_nz^n $的和函数f,在$ D_O(0,R) $上连续。对于复数的情况,还有$\forall 0<r<R$
	\[a_n=\frac{1}{2\pi r^n}\int_{0}^{2\pi}f(re^{i\theta})e^{-in\theta} d\theta \]
\end{Cor}

\begin{Prop}
	$\mathbb{K}=\mathbb{R} $时,$ \sum a_nt^n $可逐项求导、逐项积分
\end{Prop}

\begin{Def}[$ f $的Taylor级数]
	$ \sum \frac{1}{k!}f^{(k)}(0)x^k $称为Taylor级数
\end{Def}
\begin{Remark}
	Taylor级数未必收敛,就算收敛和函数也未必等于$ f $
\end{Remark}

\begin{Thm}[Borel引理]
	任给实数列$ (a_n)_{n\in\mathbb{N}}$,存在一个$ (0,1) $上的光滑函数$ f $,使得$ \frac{1}{n!}f^{(n)}(0)=a_n $
\end{Thm}

\begin{Def}[复解析]
	设$ f:U\to C,\exists r>0$ 和幂级数$ \sum a_n(z-z_0)^n $使得$ \forall z\in D_O(z_0,r),f(z)=\sum\limits a_n(z-z_0)^n $
\end{Def}

\begin{Thm}[幂级数展开的唯一性]
	设$ \sum a_nz^n $,$ \sum b_nz^n $两个在0附近收敛的幂级数,若有$ z_i\to 0 $使$ \sum\limits_{n=0}^{+\infty}a_nz_i^n $
\end{Thm}


\begin{Thm}
	f在$ (-a,a) $上$ C^\infty $,若存在$ \rho >0 ,M\in\mathbb{R}^+$,使$ \forall x\in(-\rho,\rho),\forall n\in\mathbb{N} $,$|f^{(n)}|\leqslant \frac{M}{\rho^n}n!$,则f在0点的Taylor级数收敛于f
\end{Thm}

\section{广义积分(intégrale généralisée)}
\[ \textrm{反常积分}\langle\begin{array}{ccc}\textrm{区间无界}&\to&\textrm{广义积分}\\
\textrm{函数无界}&\to&\textrm{瑕积分}\end{array} \]

\begin{Def}[在$ [a,+\infty) $上的积分]
	$ \int_{a}^{+\infty}f=\lim\limits_{x\to\infty}\int_{a}^{x}f	 $
\end{Def}

\begin{Prop}[简单的性质]
	\hspace*{\fill}
	\begin{itemize}
		\item $ \int_{a}^{\infty}f $收敛性与a无关
		\item $ f\geqslant 0 \implies \int_{a}^{\infty}f\geqslant0$
		\item 单调有界则收敛
		\item 正值函数可用比较判别法
	\end{itemize}
\end{Prop}

\begin{Def}[可积]
	称$ f $可积,若$ \int_{a}^{\infty}|f| $收敛
\end{Def}

\begin{Def}[在$ [a,b) $上的积分]
	$ \int_{a}^{b}f= \lim\limits_{x\to b}\int_{a}^{x}f $
\end{Def}
\begin{Prop}[简单的性质]
	不作赘述
	\hspace*{\fill}
	\begin{itemize}
		\item 线性性
		\item 区间可加性
		\item 符号相容性
		\item 分部积分
		\item 变量代换
	\end{itemize}
\end{Prop}

\begin{Thm}[Dirichlet判别法]
	$ x\to\int_{a}^{x}f $关于x有界,$ g $单调趋于0,$ \lim\limits_{x\to 0}g=0 $
	
\end{Thm}

\begin{Thm}[Abel判别法]
	$ \int_{a}^{+\infty} $收敛,g单调有界,则$ \int_{a}^{+\infty}fg $收敛
\end{Thm}

\begin{Prop}[渐近公式]
	
\end{Prop}

\begin{Thm}[第二中值定理,mean value]
	$ f,g \in CM([a,b],\mathbb{R}) $且g单调,则$ \exists\xi\in[a,b] $,s.t.\[ \int_{a}^b fg=g(a)\int_{a}^{\xi}f+g(b)\int_{\xi}^{b}f  \]
\end{Thm}




\section{常微分方程(ODE)}
\subsection{一阶线性方程组}

\begin{Setting}[抽象]
	\begin{align*}
	\intertext{设a,b均连续}	
		&a:I\to \mathcal{L}(E)\\
		&b:I\to E\\
	\intertext{若存在$ \varphi:I\to E $可导,且满足}
	&
	\intertext{则称$\varphi$是一阶线性方程组ODE的解}
	\end{align*}
\end{Setting}

\begin{Def}[矩阵形式]
	\[  \begin{pmatrix}x_1\\x_2\\\vdots\\x_n\end{pmatrix}'=A(t)\begin{pmatrix}x_1\\x_2\\\vdots\\x_n\end{pmatrix}+B(t)   \]
\end{Def}
\begin{Prop}[解空间结构]
	对于方程
	\[X'(t)=A(t)X(t)+B(t)  \eqno(S)  \]
	\[X'(t)=A(t)X(t)  \eqno(S_0)  \]
	$ S_0 $的解是一个向量空间,S的解是一个仿射子空间,$ S=\varphi_p+S_0 $
	
\end{Prop}
\begin{Def}[初值问题,Cauchy问题]
	\[\begin{cases}
		\varphi'(t)=a(t)\varphi(t)+b(t)\\
		x(t_0)=x_0
	\end{cases}\]
\end{Def}

\begin{Thm}
	一阶线性ODE system Cauchy问题解的存在唯一性(用迭代法证明)
\end{Thm}

\begin{Thm}[常数变易法]
	已知$ X'(t)=A(t)X(t)  $的一组解是$(\varphi_1,\cdots,\varphi_n) $
	则$ (\varphi_1,\cdots,\varphi_n) $线性无关

	令$ \varphi=\sum\limits_{i=1}^n\lambda_i(t)\varphi_i(t) $,代入方程得,$ \sum\limits_{i=1}^n\lambda'_i(t)\varphi(t)=B(t) $
	逐一解出$ \lambda_i $可知非齐次方程的解。
\end{Thm}

\subsection{线性二阶方程}
\begin{Setting}
	\[X''(t)+a_1(t)X'(t)+a_0(t)X(t)=b(t)  \eqno{(E)} \]
	\[X''(t)+a_1(t)X'(t)+a_0(t)X(t)=0  \eqno{(E_0)} \]
\end{Setting}
\begin{Def}[Wronskien行列式]
	设$ \varphi_1,\varphi_2 $是$ (E_0) $得两个解,定义Wronskien行列式为
	\[ W_{\varphi_1,\varphi_2}(t)=det\begin{pmatrix}
	\varphi_1(t) & \varphi_2(t)\\
	\varphi_1'(t) & \varphi'_2(t)
	\end{pmatrix}   \]
\end{Def}
\begin{Prop}
	由$ (E_0) $可得,$ W'+a_1(t)W=0 $
\end{Prop}
\begin{Remark}
	由上可知,W一旦在一点为0,则全为0
\end{Remark}
\begin{Remark}
	由W可求,故若知一解,则知另一解。
\end{Remark}


\begin{Prop}[如何求第一个解]
	\hspace*{\fill}
	\begin{itemize}
		\item 找简单函数
		\item 幂级数法
		\item 变量代换法
	\end{itemize}
\end{Prop}


\section{NVS值实变函数的微分和积分}

\begin{Def}[极限]
	$f:I\to E  $,则$\tau_a(f):\begin{array}[t]{ccc}I\setminus\{a\}&\longrightarrow &E\\t&\longmapsto&\frac{f(t)-f(a)}{t-a}\end{array}  $\\
	称f在a点可导,若$ \tau_a(f) $在a点有极限,记为$ f'(a)=Df(a)=\lim\limits_{a}\tau_a(f) $
\end{Def}

\begin{Thm}[有限增量不等式]
	\[|f(b)-f(a)|\leqslant C\cdot sup\|f'\|\cdot(b-a)    \]
	其中C依赖于E的维数
\end{Thm}

\begin{Remark}
	微分中值定理、Language定理,有限增量定理都因为E中没有大小关系而失效
\end{Remark}

\begin{Thm}[逼近定理]
	$ \forall f\in CM([a,b],E) ,\forall\varepsilon>0,\exists $ a step function g,\[ sup||f-g||<\varepsilon \]
\end{Thm}

\begin{Def}
	\[\int\limits_{[a,b]}f=\lim\limits_{n\to\infty}\int g_n   \]
\end{Def}



\section{Riemann积分}
\subsection{积分理论}
\begin{Def}[Riemann可积]
	称$ f $ 是Riemann可积的,若$ \exists l\in E ,\forall\varepsilon>0,\exists\delta>0\forall[a,b] $的分割$ \pi:a=t_0<t_1<\cdots<t_n=b ,\|\pi\|=max{|t_i-t_{i-1}|}\leqslant\delta,\forall \xi_i\in[t_i,t_{i-1}]$有
	\[  \|  \sum\limits_{i=1}^nf(\xi_i)(t_i-t_{i-1})  \|<\varepsilon  \]
\end{Def}

\begin{Def}
	\begin{align*}
		\underline{S}(f,\pi)=\sum\limits_{i=1}^n \sup\limits_{[x_{i-1},x_i]}f\cdot |x_i-x_{i-1}|   \\
		\overline{S}(f,\pi)=\sum\limits_{i=1}^n \inf\limits_{[x_{i-1},x_i]}f\cdot |x_i-x_{i-1}| 
	\end{align*}
	
	\begin{align*}
		\overline{I}=\sup\limits_{\pi}\underline{S}(f,\pi)   \\
		\underline{I}=\inf\limits_{\pi}\overline{S}(f,\pi) 
	\end{align*}
\end{Def}
\begin{Lem}
	\[ \underline{I}=\sup\limits_{\pi}\underline{S}(f,\pi)=\lim\limits_{||\pi||\to0}\underline{S}(f,\pi)   \]
\end{Lem}
\begin{Lem}
	$ f:[a,b]\to\mathbb{R},W=\sup f-\inf f ,\pi'$比$ \pi $细k个点,则
	\[  \underline{S}(f,\pi')\leqslant\underline{S}(f,\pi)+kW||\pi||   \]
\end{Lem}

\begin{Thm}
	f有界,以下等价
	\begin{description}
		\item[(1)]$ f:[a,b]\to\mathbb{R} $是Riemann可积 
		\item[(2)]$\lim\limits_{||\pi||\to0}(\sum\omega_i\Delta x_i)=0$
		\item[(3)]$ \overline{I}=\underline{I} $
	\end{description}
\end{Thm}

\begin{Def}[零测]
	$ A\subset \mathbb{R} $称为是零测的,
	若$ \forall\varepsilon>0 $,
	$ \exists $一列开区间 $ ((a_i,b_i))_{i\in\mathbb{N}} $,
	使$ A\subset \bigcup\limits_{i=1}^{\infty}(a_i,b_i) $且$\sum |b_i-a_i|<\varepsilon $ 
\end{Def}
\begin{Def}[连续点]
	定义\[ \omega(x,r)=\sup\limits_{I\cap B(x,r)}f-\inf \limits_{I\cap B(x,r)}f\]
	记\[ \omega(x)=\lim\limits_{r\to0^+}\omega(x,r)  \]
	x是f的连续点等且仅当$ \omega(x)=0 $
\end{Def}

\begin{Thm}[Lebesque定理]
	$ f $Riemann可积$\Leftrightarrow$f的间断点集是零测的
\end{Thm}

\begin{Thm}
	$ f_n:[a,b]\to\mathbb{R} $ Riemann可积,$ f_n\rightrightarrows f $,则$ f $也Riemann可积,且$ \int_{a}^{b}|f_n-f|\to 0 $
\end{Thm}

\begin{Def}[一致可积]
	称$ f_n:[a,b]\to\mathbb{R} $一致可积,若$ \forall\epsilon>0\exists\delta>0,\forall||\pi||<\delta,\forall\xi_i\in[x_{i-1},x_i] $有\[ |\sum f_n(\xi_i)\Delta x_i-\int_{a}^{b}f_n|<\epsilon \]
\end{Def}
\begin{Prop}
	等度连续$ \implies $一致可积
\end{Prop}
\subsection{关于参数}

\begin{Thm}[关于参数的积分]
	若f连续(条件过强),则
	\[\int_{c}^{d}\int_{a}^{b}f(x,y)dxdy=\int_{a}^{b}\int_{c}^{d}f(x,y)dydx \]
\end{Thm}

\begin{Thm}[参数积分]
	设$ f $,$ \frac{\partial f}{\partial y} $均在$ [a,b]\times[c,d] $上连续(条件过强),则$ \varphi(y)=\int_{a}^{b}f(x,y)dx $关于y可导,且\[ \varphi'(y)=\int_{a}^{b}\frac{\partial f}{\partial y}dx \]
\end{Thm}

\begin{Thm}[含参数的上下限积分]
		设$ f $,$ \frac{\partial f}{\partial y} $均在$ (a,b)\times[c,d] $上连续(条件过强)
		$ u,v $在[c,d]$\to(a,b)$可导,则
	\[ \varphi'(y)=v'(y)f(v(y),y)-u'(y)f(u(y),y)+\int_{u(y
		)}^{v(y)}\frac{\partial f}{\partial y}(x,y)dx   \]
\end{Thm}

\subsection{卷积}
\begin{Def}[卷积]
	$ \varphi:\mathbb{R}\to\mathbb{R} $,$ \varphi $光滑,当$ |x|>1 ,\varphi(x)=0$,$ f:\mathbb{R}\to\mathbb{R} $连续,令\[ g(y)=\int\limits_\mathbb{R} f(x)\varphi(y-x)   \]
\end{Def}



\begin{Thm}
	设$ f:\mathbb{R}\to\mathbb{R} $连续,$ \varphi_\epsilon=\frac{1}{\epsilon}\varphi(\frac{x}{\epsilon}) $,
	令\[ g_\epsilon(y)=\int_{\mathbb{R}}f(x)\varphi_\epsilon(y-x) dx   \]
	则$ \forall x\in \mathbb{R} ,\lim\limits_{\epsilon\to0}g_\epsilon(x)=f(x)    $
\end{Thm}


\subsection{含参变量的反常积分}

\begin{Setting}
	\[ f:[a,+\infty)\times[\alpha,\beta]\to\mathbb{R} \]
	\[ \varphi(u)=\int_{a}^{\infty}f(x,u)dx  \]
\end{Setting}

\begin{Def}[一致收敛]
	若$ \forall\epsilon>0,\exists A_0>0,\forall u\in[\alpha,\beta] ,\forall A>A_0$
	\[  |\int_{A}^{\infty}f(x,u)dx |\leqslant\epsilon  \]
	则称$ \int_{A}^{\infty}f(x,u)dx $关于$ u\in[\alpha,\beta] $一致收敛
\end{Def}

\begin{Prop}[等价的判定]
	\[ \eta(A)=\sup\limits_{u\in[\alpha,\beta]}\to 0 (A\to \infty)  \]
\end{Prop}

\begin{Prop}[Cauchy]
	$ \forall\epsilon>0,\exists A_0>0,\forall u\in[\alpha,\beta],\forall A_1,A_2>A_0 $
	\[  |\int_{A_1}^{A_2}f(x,u)dx |<\epsilon  \]
\end{Prop}


\begin{Prop}[Weierstrass判别法]
	若存在非负的$ F(x) $,使$ \int_{a}^{\infty}F(x)dx $收敛,且
	\[ \sup\limits_{[\alpha,\beta]}(f(x,u))<F(x) \]
\end{Prop}


\begin{Thm}[Abel判别法]
	$ \int_{a}^{\infty}f(x,u)dx $一致收敛,$ g(x,u) $作为x的函数单调,$ g(x,u) $关于u,x一致有界,
	则$ \int_{a}^{\infty}(fg)(u,x)dx $一致收敛
\end{Thm}

\begin{Thm}[Dirichlet判别法]
	$ \int_{a}^{A}f(x,u)dx $关于A,u一致有界,$ g(x,u) $固定u作为x的函数单调,当$ x\to \infty $时,g关于u一致趋于0,
	则$ \int_{a}^{\infty}(fg)(u,x)dx $一致收敛
\end{Thm}

\begin{Thm}
	$ f_n,f:[a,\infty)\to\mathbb{R}$,$ \forall M>0,f_n\rightrightarrows f  $ on $ [a,M] $,且广义积分$\int_{a}^{+\infty}f_n  $一致收敛,则$ \lim\limits_{n\to\infty}\int_{a}^{\infty}f_ndx=\int_{a}^{\infty}fdx $
\end{Thm}

\begin{Prop}[对参数积分]
	$ \varphi(u):=\int_{a}^{+\infty}f(x,u)dx $关于u一致收敛,则$ \int_{\alpha}^{\beta}\varphi(u)du=\int_{a}^{\infty}(\int_{\alpha}^{\beta}f(x,u)du)dx $
\end{Prop}

\begin{Prop}[对参数求导]
	$ f,\frac{\partial f}{\partial u}$
	连续,$\int_{a}^{\infty}f(x,u)dx$存在,
	$\int_{a}^{\infty}\frac{\partial f}{\partial u}dx  $关于u一致收敛,则$ \varphi(u) $可导,且$ \varphi'(u)=\int_{a}^{\infty}\frac{\partial f}{\partial u}dx $
\end{Prop}

\begin{Thm}
	\begin{align*}
		&1.f\textrm{在}[a,\infty)\times[\alpha,\infty)\textrm{连续}\\
		&2.\int_{a}^{\infty}f(x,u)dx\textrm{关于u内闭一致收敛}\\
		&3.\int_{\alpha}^{\infty}f(x,u)dx\textrm{关于x内闭一致收敛}\\
		&4.\int_{a}^{\infty}\int_{\alpha}^{\infty}|f(x,u)|dudx\textrm{存在}
	\end{align*}
\end{Thm}
\begin{Remark}
	“一致性”解决问题的方法随着维数的增加愈发复杂,造成了方法的崩溃,需要其他方法\textbf{控制收敛定理}来解决
\end{Remark}


\section{习题}
\subsection{}\noindent
1.设A为$ n\times n $的实对称矩阵(A=$ A^T $)

(1)A的特征值为实数

(2)A可对角化

2.设$ A,B\in M_n(\mathbb{R}) $

$\left.a\right)$:设A,B的特征值为$ \lambda_1(A)\ge\lambda_2(A)\ge\cdots\ge\lambda_n(A)\quad\lambda_1(B)\ge\lambda_2(B)\ge\cdots\ge\lambda_n(B) $

设$ B\ge A\quad (x^TBx\ge x^TAx,\forall x\in \mathbb{R})$,求证:$ \lambda_k(A)\ge\lambda_k(A)\ \forall k $

$\left.b\right)$:设C是对称实矩阵,若$c_{i,i}\ge\sum\limits_{j\not=i}\vert c_{i,j}\vert \ \forall i\quad $求证:$\forall X\in R^n\  X^TCX\ge 0 $

$\left.c\right)$:定义$=max\vert c_{i,j}\vert$,若$ \Vert A-B\Vert\le \frac{\epsilon}{n} $求证:$ \epsilon I\pm (A-B)\ge 0 $

$\left.d\right)$:求证:$ \lambda_k(A) $关于A是Lipschtz的,即$\vert \lambda_k(A)-\lambda_k(B)\vert\le n\vert A-B \vert $

$\left.e\right)$:$ \lambda_k(A) $关于A是否可微?

\subsection{}\noindent
1.(1):在$ M_n(\mathbb{C}) $中,可对角化矩阵全体稠密

(2):在$ M_n(\mathbb{R}) $中不对

2.设C为赋范向量空间E中的非空紧集,$ f:E\to E $是一个连续的自同态,且$ f(C)\subset C $

$\left.a\right)$:令$ f_n=\frac{1}{n}\sum\limits_{k=0}^{n-1}f^{(k)} $,求证$ f_n $为连续自同态

$\left.b\right)$:求证:存在$ M>0\ s.t.\ \forall y\in f_n(C)  $有$ \Vert f(y)-y\Vert\le \frac{2M}{n} $

$\left.c\right)$:证明f中至少有一个不动点

3.设
\subsection{}\noindent
1.设$ (f_n)_{n\in\mathbb{N}}$为有界区间I上取值于有界区间J的连续函数列,且一致收敛,$ \varphi:J\to R $连续

$\left.a\right)$:$ (\varphi\circ f_n)_{n\in\mathbb{N}} $不一定一致连续

$\left.b\right)$:若J为闭区间,则$ (\varphi\circ f_n)_{n\in\mathbb{N}} $一致连续

2.\textbf{Dini第一定理}:设K为有限维向量空间中的一个紧集,$ (f_n)_{n\in \mathbb{N}} $是一个递增的函数列,其中$ f_n\in C(K,\mathbb{R}) $,
证明:若$ (f_n)_{n\in \mathbb{N}} $收敛于$ f $,则此收敛为一致收敛

3.\textbf{Dini第二定理}:设$ (f_n)_{n\in \mathbb{N}} $是一个[a,b]上的实函数列,且收敛于[a,b]上连续的函数$ f $,设$ f_n$在[a,b]上递增,证明$ (f_n)_{n\in \mathbb{N}} $一致收敛到$ f $


\subsection{}\noindent
1.设$E \subset \mathbb{R} $可数,$ f_n:E\to \mathbb{R} \quad \vert f_n(x)\vert<1\ \forall x\in E$

求证:存在子列$ f_{n_k}\to f \ in\ E$简单收敛

2.设$(f_n)$为$ [0,1] $上有界函数列,且对任意的n,$ f_n $在$ [0,1] $上单调递增

求证:存在简单收敛的子列

3.设$ f $为在$ \mathbb{R} $连续的、以1为周期的函数,$ a\in (0,1) $

$\left.a\right) $求证:$ g(x)=\sum\limits_{n=0}^\infty a^nf(2^n a) $也是连续的周期为1的函数

$\left.b\right) $设$ f $为Lipschitz,即$ \exists k $,$ \vert f(x)-f(y)\vert < k\vert x-y\vert $ 

求证:$ g $为Holder,即$ \exists\alpha\in(0,1] ,C\in \mathbb{R}^+$,s.t.$ \vert g(x)-g(y)\vert < C\vert x-y\vert^\alpha  $

\subsection{}\noindent
1.Blashke selection Thm


Hausdorff distance $\quad X,Y\subset\mathbb{R}^n \quad d_H(X,Y)=\inf\{\epsilon> 0,X\subseteq Y_\epsilon\ Y\subseteq X_\epsilon \} $,其中$ X_\epsilon=\bigcup\limits_{x\in X} B(x,\epsilon) $

给定$ 0<r<R<+\infty\quad$,设$ D_k$为$ \mathbb{R}^n $中的一列凸集,且$ \exists z_k\in D_k\ $有界, $ s.t.\ B_r(z_k)\subset D_k\subset B_R(z_k) $

求证:存在$ D_k $的子列$ D_{k_j}\to D $,即$ d_H(D_{k_j},D)\to 0(j\to\infty) $

2.矩阵的收敛

$ A_k=(a_{i,j}^k)_{1\le i,j\le n}\ \ A=(a_{i,j}) \in M_n(\mathbb{K})$

 Def:$ A_k\to A\Leftrightarrow a_{i,j}^k\to a_{i,j}(k\to \infty) \ \forall i,j $

 $ A_k=(a_{i,j}^k(t))_{1\le i,j\le n}\ \ A=(a_{i,j}(t)) t\in [a,b]$

 Def:$ A_k\rightrightarrows A\Leftrightarrow a_{i,j}^k(t)\rightrightarrows a_{i,j}(t)(k\to \infty) \ \forall i,j $

Einstein notation

 $ A(t)=(a_{i,j(t)}) ,A^{-1}(t)=(a^{i,j}(t))$

 证明:$ \frac{d\ a^{i,j}(t)}{dt}=\sum\limits_{k,s}-a^{i,k}\frac{d\ a_{k,s}}{dt}a^{s,j} $即$ (A^{-1})'=-(A^{-1})\cdot A'\cdot A^{-1} $

3.$ A,B\in M_n(\mathbb{R}) $

 定义$ e^A=I+A+\frac{1}{2!}A^2\cdots\quad $验证此定义的合理性

 求证:$ \frac{d}{dt}(e^{A+tB})(0)=\int_{0}^{1}e^{(1-s)A}Be^{sA}ds $

4.设$ \sum A_k\ \sum B_k $一致收敛,求证:$ \bigg(\sum\limits_{k=0}^\infty A_k \bigg)D\bigg(\sum\limits_{\ell=0}^\infty B_\ell \bigg)=\sum\limits_{m=0}^\infty\sum\limits_{k+l=m}A_kDB_l $




\end{document}
